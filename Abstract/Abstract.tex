\documentclass[../Thesis-IJspeert.tex]{subfiles}

\begin{document}

\newlength{\smalldrop}% for my convenience
\cleardoublepage
\thispagestyle{empty}
\smalldrop=0.5\baselineskip

\begin{center}

\rule[\baselineskip]{\textwidth}{1pt}\\[-0.5\baselineskip]

{\Large \textsc{Single-Atom Trapping with Optical Tweezers for Deterministic Loading of Optical Cavities}}\\[0.5\baselineskip]

\rule[\baselineskip]{0.8\textwidth}{0.5pt}\\[-\baselineskip]

Mark IJspeert, Christ Church College\\
Department of Physics\\[0.3\baselineskip]

Thesis submitted for the degree of Doctor of Philosophy\\at the University of Oxford, Trinity 2021.\\[0.5\baselineskip]

\rule[\baselineskip]{\textwidth}{1pt}\\[-0.3\baselineskip]
\vspace{2em}
{\Large Abstract}\\[0.5\baselineskip]

\end{center}

%For truly scalable quantum computation the challenge of efficiently preparing and connecting multiple qubits must be addressed.  Proposed architectures include linear optical quantum computation and quantum networks of stationary processing nodes interlinked by flying qubits.  A quantum emitter strongly coupled to a single mode of the electric field is a versatile tool for such schemes, with the potential to provide either the on-demand generation of single-photons necessary for all-optical computation or the matter-light interface at the nodes of a quantum network.  This thesis presents the construction and characterisation of an \emph{a priori} deterministic single-photon source using single \Rb{} atoms coupled to a high finesse optical cavity that provides unparalleled control over the quantum state of the photons.  These photons demonstrate a high degree of indistinguishability and long coherence times with a two-photon Hong-Ou-Mandel visibility of \SI{70.8\pm4.6}{\percent} over the entire interaction time of \SI{300}{\ns} long photons, further increasing to $\geq\SI{97.8}{\percent}$ by post-selecting only detection events within \SI{23}{\ns} of each other.  No degradation of performance is observed when interfering these pairs through a \NbyN{4}{4} multimode interferometer integrated onto a photonic chip with non-classical correlations measured between photon detections orders of magnitude further separated than the propagation time through the chip.  The production scheme for polarised single-photons is also considered in greater detail than can be found in previous work.  Cavity birefringence is found to alter the polarisation state of the emitted photons and a novel model to incorporate this effect into the coupling of an atom-cavity system is presented.  Further deviations from a simple coupling model are observed in the breakdown of the hyperfine atomic structure in the intermediate strength magnetic fields required for polarised photon emission.  The behaviour of the system in this regime is modelled and shown to explain previously unresolved observations of the asymmetric production efficiencies of orthogonally polarised photons.
Reconfigurable arrays of trapped single atoms are an excellent platform for the simulation of many-body physics and the realisation of high-fidelity quantum gates. The confinement of atoms is often achieved with focussed laser beams acting as optical dipole-force traps that allow for both static and dynamic positioning of atoms. In these traps, light-assisted collisions---enhancing the two-atom loss rate---ensure that single atom occupation of traps can be realised. However, the time-averaged probability of trapping a single atom is limited to $0.5$ when loading directly from a surrounding cloud of laser-cooled atoms, preventing deterministic filling of large arrays. In this work, we demonstrate that increasing the depth of a static, optical dipole trap enables the transition from fast loading on a timescale of \SI{2.1}{\second} to an extended trap lifetime of \SI{7.9}{\second}. This method demonstrates an achievable filling ratio of $\SI{79\pm2}{\percent}$ without the need of rearranging atoms to fill vacant traps.
\pagebreak

\end{document}